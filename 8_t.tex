\documentclass[11pt,a4paper, fqlen]{article}
\usepackage[utf8x]{inputenc}
\usepackage[T1]{fontenc}
\usepackage[russian]{babel}
\usepackage{ucs}
\usepackage{amsmath}
\usepackage{amsfonts}
\usepackage{amssymb}
\usepackage{graphicx}
\usepackage{fullpage}
\usepackage[margin=1mm]{geometry}


\begin{document}
	\section*{Билет 8.\\ Математическая постановка задач для уравнения теплопроводности.}

	
	\section{Задача в общем виде}
	Решаем задачу на стержне, для начала конечном. (На отрезке [0,l]) \\
	
	Какие физические законы знаем?
	
	 \textbf {Закон Фурье} \\
	 
	 
		Если распределение температоры неравномерно, то возникают потоки тепла из мест высокой в места низкой температуры.
		\begin{flalign}
		 Q = -k * \frac{u_2 - u_1}{l_2 - l_1} * S = -k * \frac{\partial u}{\partial x} S \\
		dQ = -k * \frac{\partial u}{\partial x} S dt \\
		Q = -S \int_{t1}^{t2} k \frac{\partial u}{\partial x} dt 
		\end{flalign}
		
		

	
	
	Итак, уравнение теплопроводности: \\
	Начальные условия: $u(x,0) = \varphi(x)$ \\
	Граничные условия: \\
	1. Условия 1 рода 
	
	 $\quad u(0,t) = \mu_1(t)$ 
	 
	 $\quad u(l,t) = \mu_2(t)$  \\ 
	2. Условие 2 рода (теплообмен)
	
	$\quad u_x(0,t) = \nu_1(t)$ 
	
	$\quad u_x(l,t) = \nu_2(t)$
	
	Если $u_x(0,t) = 0$, то конец теплоизолирован \\
	Почему? По закону Фурье: $\frac{du}{dx} = 0 => Q(x=0) = 0 $
	=> тепло не уходит и не приходит в этой точке.
	
	\section{Постановка конкретных задач}
	\begin{enumerate}
		\item Задача с неоднородным условием
		\noindent
		\begin{flalign*}
		&u_t = a^2u_{xx} + f& \\
		&u(0,t) = \mu_1(t)& \\
		&u(l,t) = \mu_2(t)& \\
		&u(x,0) = \varphi(x)&
		\end{flalign*}
		
		\item Задача на бесконечной прямой ($-\infty < x < \infty$)
		\begin{flalign*}
		&u_t = a^2u_{xx} & \\
		&u(x,0) = \varphi(x)& \\
		&|u(x,t)|\ \text{ограничен}&
		\end{flalign*}
		
		\item Задача на полубесконечной прямой ($0 < x < \infty$)
		\begin{flalign*}
		&u_t = a^2u_{xx} & \\
		&u(x,0) = \varphi(x)& \\
		&u_x(0,t) = 0\ \text{или} \ u(0,t) = 0& \\
		&|u(x,t)|\ \text{ограничен}&
		\end{flalign*}
	\end{enumerate}

\section{Третья формула Грина}
Пусть $\Sigma$ - область на плоскости, L - ее граница. Функции $P(x,y), Q(x,y)$ непрерывны в этой области вместе со своими частными производными. Тогда выполнена формула Грина:
$$
\oint \limits_{L} Pdx + Qdy = \iint \limits_{\Sigma} (\frac{\partial Q}{\partial x} - \frac{\partial P}{\partial y})dxdy
$$
\end{document}