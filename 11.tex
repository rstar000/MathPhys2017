\documentclass[11pt,a4paper, fqlen]{article}
\usepackage[utf8x]{inputenc}
\usepackage[T1]{fontenc}
\usepackage[russian]{babel}
\usepackage{ucs}
\usepackage{amsmath}
\usepackage{amsfonts}
\usepackage{amssymb}
\usepackage{graphicx}
\usepackage[margin=1mm]{geometry}


\usepackage{amsthm}

\title{Билет 11. Существование  решения задачи Коши для уравнения теплопроводности}
\date{}

\newtheorem*{thdef}{Теорема}
\begin{document}
	\section*{Билет 11. \\Существование  решения задачи Коши для уравнения теплопроводности}
	\section{Постановка задачи}
	\begin{gather}
	\begin{cases}
	u_t = a^2 u_{xx}, \quad -\infty < x < \infty\\
	u(x,0) = \varphi(x)
	\end{cases}
	\end{gather}
	
	Хотим доказать, что решение $u(x,t)$ существует и единственно.
	
	\section{Подготовка}
	
	Используем метод разделения переменных. $u(x,t) = X(x)T(t)$
	
	Подставим это в задачу. Получим $T'X = a^2X''T$. \\
	Домножим всё на $\frac{1}{a^2XT}$.
	$$\frac{X''}{X} = \frac{T'}{a^2 T} = -\lambda^2, \quad \lambda = const > 0$$
	
	В итоге получили такую систему:
	$$
	\begin{cases}
	X'' + \lambda^2X = 0 \\
	T' + a^2\lambda^2T = 0
	\end{cases}
	$$
	
	Решения системы: 
	\[
	\begin{cases}
	X(x) = exp(i \lambda x)\\
	T(t) = exp(-a^2 \lambda^2 t)
	\end{cases}
	\]
	
	Итак, решение исходной системы: $u(x,t) = exp(i \lambda x - a^2 \lambda^2 t)$
	Пусть $A(\lambda)$ - некоторая функция. Тогда $u_\lambda = A(\lambda)  u(x,t)$ - тоже решение системы(функция по сути константа).
	
	Определим решение таким образом:
	\[
	u(x,t) = \int \limits_{-\infty}^{\infty} A(\lambda) exp(i \lambda x - a^2 \lambda^2 t) d\lambda
	\]
	
	Сделаем так, что она удовлетворяет начальному условию $u(x,0) = \varphi(x)$
	\begin{gather}
	\varphi (x) = \int \limits_{-\infty}^{\infty} A(\lambda) e^{i\lambda x} d\lambda \\	
	A(\lambda) = \frac{1}{2\pi}  \int \limits_{-\infty}^{\infty} e^{-i \lambda s} \varphi(s) ds \quad \text{//Коэффициенты ряда Фурье}
	\end{gather}
	
	Подставим найденные коэффициенты A в решение:
	$$
	u(x,t) = \int \limits_{-\infty}^{\infty} \frac{1}{2\pi} \left[\int \limits_{-\infty}^{\infty} e^{-i \lambda s} \varphi(s) ds \right] exp(i \lambda x - a^2 \lambda^2 t) d\lambda = 
	\int \limits_{-\infty}^{\infty} \varphi(s) \left[\int \limits_{-\infty}^{\infty} \frac{1}{2\pi} exp(i \lambda (x-s) - a^2 \lambda^2 t) d\lambda \right] ds
	$$
	Внутренний интеграл вполне можно посчитать. В результате получится:
	$$
	u(x,t) = \frac{1}{\sqrt{4\pi a^2 t}} \int \limits_{-\infty}^{\infty} exp(-\frac{(x-s)^2}{4a^2t}) \varphi(s)ds
	$$
	Более короткая запись:
	\begin{gather}
	G(x,s,t) = \frac{1}{\sqrt{4\pi a^2 t}} exp(-\frac{(x-s)^2}{4a^2t}) \\	
	u(x,t) = \int \limits_{-\infty}^{\infty} G(x,s,t) \varphi(s) ds
	\end{gather}
	
	\section{Теорема}
	\begin{thdef}
		Пусть $\varphi(x)$ - начальное условие, такое что оно непрерывно по x и ограничено. Тогда (5) непрерывно по x,t и удовлетворяет уравнению теплопроводности при $x \in \Re, t>0$. К тому же, $\lim\limits_{t->0+} u(x,t) = \varphi(x)$
	\end{thdef}
	
	\begin{proof}
		\textbf{1.} Для начала докажем, что u(x,t) непрерывна. Для этого достаточно доказать, что u(x,t) непрерывна в прямоугольнике $П = \{(x,t): -L < x < L, t_0 < t < t\}$ где все эти пределы - const > 0. \\
		Все функции в интеграле из u (а это G и $\varphi$) непрерывны в П. Тогда если интеграл сходится равномерно, то и u тоже.
		
		\textbf{2.} Сделаем такую замену: $\frac{x-s}{2a\sqrt{t}} = \psi$. Тогда $s = x + \psi 2 a \sqrt{t}$, $ds = 2a\sqrt{t}d\psi$. Интеграл перепишем таким образом:
		
		$$
		u(x,t) = \frac{1}{\sqrt{\pi}} \int \limits_{-\infty}^{\infty} e^{-\psi^2} \phi(x+2a\psi \sqrt{t}) d\psi
		$$
		
		Если в этой формуле положить $t = 0$, то получится начальное условие (тут интеграл Пуассона). $\phi$ ограничена, => интеграл сходится равномерно. Доказан пункт про начальные условия.
		
		\textbf{3.} Теперь докажем возможнотсь дифференцирования под знаком интеграла.
		
		$$
		\frac{du}{dt} (x,t) = \int \limits_{-\infty}^{\infty} \frac{\partial G}{\partial t}(x,s,t) \phi(s) ds = [\text{делаем тут замену} \psi] = \frac{1}{\sqrt{\pi}} \int \limits_{-\infty}^{\infty} (-\frac{1}{2t} + \frac{\psi^2}{t})e^{-\psi^2} \phi(x+2a\psi \sqrt{t}) d\psi
		$$
		
		Подынтегральная функция ограничена => интеграл сходится равномерно => работает дифференцируемость по t. Точно также можно проверить и для x.
	\end{proof}
	
	
	\section{Второй вопрос}
	\textbf{Постановка внешней задачи Дирихле для уравнения Лапласа в пространстве}\\
	Пусть $\varOmega$ - некоторая открытая область в $E^3$, огранченная поверхностью $\varSigma$. 
	Задача:
	\begin{gather*}
	\Delta u = 0 \quad \text{и u непрерывна вне области} \\
	u(M) = \mu(M), \quad \mu \in \varSigma \\
	u(M) \rightrightarrows 0 \ M \rightarrow \infty 
	\end{gather*} 
\end{document}