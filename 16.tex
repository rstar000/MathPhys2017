\documentclass[11pt,a4paper]{article}
\usepackage[utf8x]{inputenc}
\usepackage[T1]{fontenc}
\usepackage[russian]{babel}
\usepackage{ucs}
\usepackage{amsmath}
\usepackage{amsfonts}
\usepackage{amssymb}
\usepackage{graphicx}
\usepackage[margin=1mm]{geometry}


\usepackage{amsthm}

\title{}
\date{}

\begin{document}
\section*{Билет 16.\\ Первая и вторая формулы Грина}
\section{Формула Остроградского}
	
Пусть поверхность $\varSigma$ состоит из конечного числа замкнутых кусков, имеющих в каждой точке касательную, причем любые прямые, параллельные прямые, параллельные осям координат, пересекают поверхность либо в конечном числе точек, либо по конечному числу отрезков.

Тогда в области $\varOmega$ для функции $\vec{A}(x,y,z) = \{P(x,y,z), Q(x,y,z), R(x,y,z)\}$, где P,Q,R - непрерывно дифференцируемые в $\bar{\varOmega}$, верна формула Остроградского-Гаусса:

$$
\iint \limits_{\varSigma} (\vec{A}, \vec{n})d\sigma = \iiint \limits_{\varOmega} div \vec{A} d\tau
$$

$$
div \vec{A} = \frac{\partial P}{\partial x} + \frac{\partial Q}{\partial y} + \frac{\partial R}{\partial z}
$$

\section{Первая формула Грина}
Пусть u,v - скалярные функции, $u,v \in C^2(\varOmega) \cap C^1{\bar{\varOmega}} $\\
$A = u * grad (v)$. Применим формулу Остроградского:
$$
\iint \limits_{\varSigma} (u * grad(v), \vec{n})d\sigma = \iiint \limits_{\varOmega} div (\vec{u * grad(v)}) d\tau
$$
Распишем левую и правую части

\textbf{1. Правая часть} \\
\begin{equation}
\begin{split}
div(u\ grad(v)) = div(u\frac{\partial v}{\partial x}, u\frac{\partial v}{\partial y}, u\frac{\partial v}{\partial z}) = \\
\frac{\partial(u\frac{\partial v}{\partial x})}{\partial x} + 
\frac{\partial(u\frac{\partial v}{\partial y})}{\partial y} +
\frac{\partial(u\frac{\partial v}{\partial z})}{\partial z} = \\
\frac{\partial u}{\partial x} \frac{\partial v}{\partial x} + 
u \frac{\partial^2 v}{\partial x^2} + 
\frac{\partial u}{\partial y} \frac{\partial v}{\partial y} + 
u \frac{\partial^2 v}{\partial y^2} + 
\frac{\partial u}{\partial z} \frac{\partial v}{\partial z} + 
u \frac{\partial^2 v}{\partial z^2} = \\
(grad(u), grad(v)) + u \cdot (
\frac{\partial^2 v}{\partial x^2} +
\frac{\partial^2 v}{\partial y^2} +
\frac{\partial^2 v}{\partial z^2}
) = \\
(grad(u), grad(v)) + u \Delta v
\end{split}
\end{equation}

\textbf{2. Левая часть} \\
\begin{equation}
\begin{split}
(grad(v), \vec{n}) = \frac{\partial v}{\partial x} \cdot n1 + \frac{\partial v}{\partial y} \cdot n2 + \frac{\partial v}{\partial z} \cdot n3 = \frac{\partial v}{\partial n}
\end{split}
\end{equation}

Итак, первая формула Грина:
$$
\iiint \limits_{\varOmega} ((grad(u), grad(v)) + u\Delta v)dr = \iint \limits_{\Sigma} u  \frac{\partial v}{\partial n} d\sigma
$$

\section{Вторая формула Грина}
Поменяем местами u и v и вычтем из исходной формулы Грина. Получим:
$$
\iiint \limits_{\varOmega} ((grad(u), grad(v)) - (grad(u), grad(v)) + u\Delta v -  v\Delta u)dr = \iint \limits_{\Sigma} u  (\frac{\partial v}{\partial n} -  v  \frac{\partial u}{\partial n}) d\sigma
$$

$$
\iiint \limits_{\varOmega} (u\Delta v -  v\Delta u)dr = \iint \limits_{\Sigma} u  (\frac{\partial v}{\partial n} -  v  \frac{\partial u}{\partial n}) d\sigma
$$

\section{Формула Д'аламбера}
Задача Коши для уравнения колебаний на прямой:
$$
\begin{cases}
	u_tt = a^2t_xx, \quad x \in \Re, t > 0\\
	u(x,0) = \varphi(x) \\
	u_t(x,0) = \psi(x)
\end{cases}
$$
Формула Даламбера
$$
u(x,t) = \frac{1}{2}(\varphi(x-at) + \varphi(x+at)) + \frac{1}{2a} \int\limits_{x-at}^{x+at} \psi(s)ds
$$
\end{document}